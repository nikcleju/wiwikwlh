\documentclass{beamer}

\let\svpar\par
\let\svitemize\itemize
\let\svenditemize\enditemize
\let\svitem\item
\def\newpar{\def\par{\svpar\vfill}}
\def\newitem{\def\item{\vfill\svitem}}
\let\svcenter\center
\let\svendcenter\endcenter
\let\svcolumn\column
\let\svendcolumn\endcolumn
\newlength\columnskip
\columnskip 0pt
\def\newcolumn{%
  \renewenvironment{column}[2]%
    {\svcolumn{##1}\setlength{\parskip}{\columnskip}##2}%
    {\svendcolumn\vspace{\columnskip}}}

\newcommand\stretchy{\only<2>{%
  \newpar\def\item{\svitem\newitem}%
  \renewenvironment{itemize}{\svitemize}{\svenditemize\newpar\par}%
  \renewenvironment{center}{\svcenter\newpar}{\svendcenter\newpar}%
  \newcolumn
}}

\begin{document}

%%%%%%%%%%%%%%%%%%%%%%%%
\begin{frame}{A few paragraphs (with\only<1>{out} stretching)}
\stretchy

As any dedicated reader can clearly see, the Ideal of practical reason is a representation of the things in themselves.

I have shown elsewhere, the phenomena should only be used as a canon for our understanding.

The paralogisms of practical reason are what first give rise to the architectonic of practical reason.
\end{frame}

%%%%%%%%%%%%%%%%%%%%%%%%
\begin{frame}{Displayed equation (with\only<1>{out} stretching)}
\stretchy

As any dedicated reader can clearly see, the Ideal of practical reason is a representation of the things in themselves.

I have shown elsewhere, the phenomena should only be used as a canon for our understanding.
\begin{equation*}
x^2 + y^2 = z^2
\end{equation*}
The paralogisms of practical reason are what first give rise to the architectonic of practical reason.

Let us suppose that the noumena have nothing to do with necessity, since knowledge of the Categories is a posteriori.
\end{frame}

%%%%%%%%%%%%%%%%%%%%%%%%
\begin{frame}{Mid-par equation (with\only<1>{out} stretching)}
\stretchy

As any dedicated reader can clearly see, the Ideal of practical reason is a representation of the things in themselves.

I have shown elsewhere,
\begin{equation*}
x^2 + y^2 = z^2
\end{equation*}
the phenomena should only be used as a canon for our understanding.

The paralogisms of practical reason are what first give rise to the architectonic of practical reason.

Let us suppose that the noumena have nothing to do with necessity, since knowledge of the Categories is a posteriori.
\end{frame}

%%%%%%%%%%%%%%%%%%%%%%%%
\begin{frame}{Items (with\only<1>{out} stretching)}
\stretchy

\begin{itemize}
\item As any dedicated reader can clearly see, the Ideal of practical reason is a representation of the things in themselves.
\item I have shown elsewhere, the phenomena should only be used as a canon for our understanding.
\item The paralogisms of practical reason are what first give rise to the architectonic of practical reason.
\end{itemize}
\end{frame}

%%%%%%%%%%%%%%%%%%%%%%%%
\begin{frame}{Items with subitems (with\only<1>{out} stretching)}
\stretchy

\begin{itemize}
\item As any dedicated reader can clearly see, the Ideal of practical reason is a representation of the things in themselves.
\begin{itemize}
\item I have shown elsewhere, the phenomena should only be used as a canon for our understanding.
\item The paralogisms of practical reason are what first give rise to the architectonic of practical reason.
\end{itemize}
\item Let us suppose that the noumena have nothing to do with necessity, since knowledge of the Categories is a posteriori.
\end{itemize}
\end{frame}

%%%%%%%%%%%%%%%%%%%%%%%%
\begin{frame}{Items and paragraphs (with\only<1>{out} stretching)}
\stretchy

As any dedicated reader can clearly see,
\begin{itemize}
\item The Ideal of practical reason is a representation of the things in themselves.
\item I have shown elsewhere, the phenomena should only be used as a canon for our understanding.
\item The paralogisms of practical reason are what first give rise to the architectonic of practical reason.
\end{itemize}
\end{frame}

%%%%%%%%%%%%%%%%%%%%%%%%
\begin{frame}{Items with equation (with\only<1>{out} stretching)}
\stretchy

\begin{itemize}
\item As any dedicated reader can clearly see, the Ideal of practical reason is a representation of the things in themselves.
\item I have shown elsewhere, the phenomena should only be used as a canon for our understanding.
\begin{equation*}
x^2 + y^2 = z^2
\end{equation*}
\item The paralogisms of practical reason are what first give rise to the architectonic of practical reason.
\end{itemize}
\end{frame}

%%%%%%%%%%%%%%%%%%%%%%%%
\begin{frame}{Items with centred text (with\only<1>{out} stretching)}
\stretchy

\begin{itemize}
\item As any dedicated reader can clearly see, the Ideal of practical reason is a representation of the things in themselves.
\item I have shown elsewhere, the phenomena should only be used as a canon for our understanding.
\begin{center}
There can be no doubt that the objects \dots
\end{center}
\item The paralogisms of practical reason are what first give rise to the architectonic of practical reason.
\end{itemize}
\end{frame}

%%%%%%%%%%%%%%%%%%%%%%%%
\begin{frame}{Columns (with\only<1>{out} stretching)}
\stretchy

\begin{columns}[c]
\column{1.5in}
As any dedicated reader can clearly see,

the Ideal of practical reason is

a representation of the things in themselves.

\column{1.5in}
\begin{itemize}
\item The paralogisms of practical reason 
\item are what first give rise to
\item the architectonic of practical reason.
\end{itemize}
\end{columns}

\end{frame}

\end{document}
